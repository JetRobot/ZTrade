\chapter{Part K: Variance Swaps}
Developer: Simon Leger

\noindent Validator: Aloke Mukherjee


%DEVELOPER WRITES THIS PART --->

\section{Requirements}

%<description of the problem being solved, relevant equations, algorithms, etc>
In this section, we use a portfolio of European options to compute
the value of a variance swap. We write methods to compute the value of such swaps.


\section{Design }

The value of a variance swap is computed according to the theoritical formula. 
There is in fact a formula to compute the value of a variance swap under 
no-arbitrage assumption which is not the case for volatility swaps for instance.

\section{Approach}
%<high-level description of the design>
To create a variance swap object, we use the OptionStrategy design
built in part A and we pass a pointer to such an object, with a
maturity and a forward price since it is all we need to compute
the value.


\section{Choices}
%<any important design choices you made, e.g. data structures, class hierarchy, algorithm, etc. and a justification for the decision>
We decided to create all information required in the constructor
of the object and store them like pointers so we have a function
getPrice() which calculates the price according to the values
stored and according to the following formula :

$$Price=\frac{2}{T}\left(\int_0^F\frac{1}{K^2}P(K)dK+\int_F^\infty \frac{1}{K^2}C(K)dK\right)$$

where $T$ is the maturity of the contract, $F$ is the forward price,
$P(K)$ is the price of the put and $C(K)$ the price of a call with
strikes $K$.


\section{Unit tests}

%<short descriptions of each subtest>
Test on the VIX index : To test the accuracy of the variance swap implementation 
we create a portfolio of puts and calls options, by using the OptionStrategy class. 
We take a minimum and a maximum strike and a step for this and we create a bench of options 
with these strikes, calls if the strike is higher than the forward value of S\&P and puts in the other case. 
With a minimum strike of 500, a maximum strike of 3500, a step of 10, we take a spot at 1200, the one month 
interest rate is around 4.3% and we assume a flat volatility of 33%, we get a price of 10.9, which is very close to 
the current value of the VIX index which is around 11.



%VALIDATOR WRITES THIS PART --->

\section{Validation}


\subsection{Approach}

%<i.e. what alternate method was used to validate the results - if this required a lot of code then similar outline as above should apply>
For this section's validation, the more efficient way to test the accuracy of the results provided bt 
the variance swaps part was to build the formula in an Excel spreadsheet built from prices of options, 
both calls and puts whose values have been calculated with the Black Scholes closed formula. We then compute 
the same price as in the example provided in the unit test part and we found the same result.
As well as for the VIX index, where we have very similar prices to exact value of the index found 
on the CBOE web site, such that we can consider that the results provided by this class are good.
