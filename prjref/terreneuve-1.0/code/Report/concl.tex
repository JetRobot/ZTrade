\chapter{Conclusion}

\par This report is quite long, so we will not spread pages as a conclusion.

We obviously learnt a lot with this project, as we have tried to share a lot on the issues we were facing on a daily basis. The number of emails of the list that were sent per day is amazingly huge. 

It leads us to the fact that the team work was excellent on this project, and the use of CVS and Skype for conference calls -- we said we thanked the nerds that invented the internet ! -- made sure that from the beginning the objects were plugging together exactly. We avoided much of the last minute pain in doing that.

We hope the project is in the most deliverable state as possible, even if the user cannot "consolely" play with all the products, the code is there anyways.

\bigskip

\textbf{Future Work}

This project clearly illustrates the complexity of the universe of financial products.  Additionally, there can be many approaches to modelling each product. In this project we have implemented some of the most popular modeling techniques including closed-forms, Monte Carlo simulation and binomial trees, and not just in C++ but also in Excel and Matlab! As developers attempting to work with this variety, the first and foremost imperative is "get it working". 
We've learned that this is not a simple task: to begin with how will you even know it is working? But once you've conquered that peak, the sky is the limit: approaches can be changed, parameters altered, models made more precise, computations made more efficient. This project has been a great experience because it has exposed us to a wide variety of approaches and techniques. And having not died from exposure, we can see from these heights how much exciting work there is left to do!

\bigskip

\bigskip

-- The Terreneuve Team that will now celebrate as a team the end of the semester.

